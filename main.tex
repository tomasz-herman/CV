\documentclass[margin,line]{res}
%\nonstopmode

\usepackage{polski}

\oddsidemargin -.5in
\evensidemargin -.5in
\textwidth=6.0in
\itemsep=0in
\parsep=0in
% if using pdflatex:
\setlength{\pdfpagewidth}{\paperwidth}
\setlength{\pdfpageheight}{\paperheight} 

\newenvironment{list1}{
  \begin{list}{\ding{113}}{%
      \setlength{\itemsep}{0in}
      \setlength{\parsep}{0in} \setlength{\parskip}{0in}
      \setlength{\topsep}{0in} \setlength{\partopsep}{0in} 
      \setlength{\leftmargin}{0.17in}}}{\end{list}}
\newenvironment{list2}{
  \begin{list}{$\bullet$}{%
      \setlength{\itemsep}{0in}
      \setlength{\parsep}{0in} \setlength{\parskip}{0in}
      \setlength{\topsep}{0in} \setlength{\partopsep}{0in} 
      \setlength{\leftmargin}{0.2in}}}{\end{list}}


\begin{document}

\name{Tomasz Herman \vspace*{.1in}}

\begin{resume}
\section{\sc Kontakt}
\vspace{.05in}
\begin{tabular}{@{}p{2in}p{3in}}
     Polska     & {\it Tel.:}  (+48) 606 159 378 \\ & {\it E-mail:} tomasz.herman@pm.me
    
\end{tabular}


\section{\sc Education}
	{\bf Politechnika Warszawska}, Warszawa, Polska\\
	%{\em}
	\vspace*{-.1in}
	\begin{list1}
	\item[] Magister, {\bf Informatyka}, z wyróżnieniem, Marzec 2021 - Październik 2022 
	\newline Stypendium rektora za dobre wyniki w nauce
	\newline spec. {\bf Projektowanie systemów CAD/CAM}
    \vspace*{.05in}
	
	\vspace*{.05in}
	\end{list1}

	{\bf Politechnika Warszawska}, Warszawa, Polska\\
	%{\em}
	\vspace*{-.1in}
	\begin{list1}
	\item[] Inżynier, {\bf Informatyka}, Październik - Marzec 2021
	\vspace*{.05in}
	\end{list1}
	

\section{\sc Doświadczenie}

    {\bf 11.2022 - obecnie} Wykonawca projektu "Surykatka" na {\bf Politechnice Warszawskiej}
    \begin{list2}
    	\item Analiza danych i wizualizacja danych z symulacji dla sił specjalnych
    	\item Technologie: C\#, Unity, OpenGL
    \end{list2}
    
    {\bf 03.2022 - obecnie} Nauczanie na {\bf Politechnice Warszawskiej}
    \begin{list2}
    	\item Prowadzenie laboratoriów z:
			\begin{list2}
				\item Programowanie w Środowisku Graficznym
				\item Programowanie 3 - Zaawansowane
				\item Tworzenie aplikacji webowych z użyciem .NET Framework
			\end{list2}
    	\item Technologie: C\#, C++, WinForms, WPF
    \end{list2}
    
	{\bf 06.2022 - 11.2022} Projekt Badawczy dla {\bf CD PROJECT RED}
	\begin{list2}
	    \item Rozwijanie narzędzi i algorytmów dla środowiska wirtualnego
	    \item Automatyzacja procesów przydatnych podczas tworzenia gier wideo
		\item Technologie: C++, Blender, Python
	\end{list2}

	{\bf 07.2020 - 05.2022} Staż/Junior Developer w {\bf Samsung R\&D Poland}
	\begin{list2}
		\item Rozwój aplikacji UI
		\item Rozwój platformy testowej dla Androida
		\item Technologie: Java, Kotlin, Android, C\#, Python
	\end{list2}


\section{\sc Umiejętności}
	
	\begin{list2}
		\item Algorytmy i struktury danych
		\item Algorytmy środowiska wirtualnego
		\item Wysoka znajomość języków {\bf Java} i {\bf C\#}
		\item Doświadczenie z {\bf C++, Python, Linux, OpenGL, Unity} and {\bf Unreal Engine}
	\end{list2}
	
\section{\sc Osiągnięcia Naukowe i Organizacyjne}
	
	\begin{list2}
		\item Wykonawca grantu rektorskiego w roku 2021 pt. InnoHaptix - Biblioteka interakcji w Wirtualnej Rzeczywistości
		\item VI Dzień Popularyzacji Matematyki, 15.09.2022, zajęcia "Eksperymentuj w wirtualnej rzeczywistości"
		\item 26 Festiwal Nauki w Warszawie, 20.09.2022, zajęcia "Eksperymentuj w wirtualnej rzeczywistości (warsztaty)"
		\item Edukacja w Wirtualnej Rzeczywistości (EduVR) - warsztaty
		\item Członek Koła Naukowego Wirtualnej Rzeczywistości
		\item 19 Targi Kół Naukowych i Organizacji Studenckich "KONIK" - Reprezentacja Koła Naukowego Wirtualnej Rzeczywistości 
	\end{list2}

\end{resume}

\end{document}




