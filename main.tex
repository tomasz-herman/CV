\documentclass[margin,line]{res}
%\nonstopmode

\usepackage{polski}

\oddsidemargin -.5in
\evensidemargin -.5in
\textwidth=6.0in
\itemsep=0in
\parsep=0in
% if using pdflatex:
\setlength{\pdfpagewidth}{\paperwidth}
\setlength{\pdfpageheight}{\paperheight} 

\newenvironment{list1}{
  \begin{list}{\ding{113}}{%
      \setlength{\itemsep}{0in}
      \setlength{\parsep}{0in} \setlength{\parskip}{0in}
      \setlength{\topsep}{0in} \setlength{\partopsep}{0in} 
      \setlength{\leftmargin}{0.17in}}}{\end{list}}
\newenvironment{list2}{
  \begin{list}{$\bullet$}{%
      \setlength{\itemsep}{0in}
      \setlength{\parsep}{0in} \setlength{\parskip}{0in}
      \setlength{\topsep}{0in} \setlength{\partopsep}{0in} 
      \setlength{\leftmargin}{0.2in}}}{\end{list}}


\begin{document}

\name{Tomasz Herman \vspace*{.1in}}

\begin{resume}
\section{\sc Kontakt}
\vspace{.05in}
\begin{tabular}{@{}p{3in}p{3in}}
     \qquad & {\it Tel.:}  (+48) 606 159 378 \\ 
     & {\it E-mail:} tomasz.herman@pw.edu.pl
    
\end{tabular}


\section{\sc Education}
	{\bf Politechnika Warszawska}, Warszawa, Polska\\
	%{\em}
	\vspace*{-.1in}
	\begin{list1}
	\item[] Magister, {\bf Informatyka i Systemy Informacyjne}, z wyróżnieniem, 
	\newline Stypendium rektora za dobre wyniki w nauce, Luty 2021 - Październik 2022 
	\newline spec. {\bf Projektowanie systemów CAD/CAM}
    \vspace*{.05in}
	
	\vspace*{.05in}
	\end{list1}

	{\bf Politechnika Warszawska}, Warszawa, Polska\\
	%{\em}
	\vspace*{-.1in}
	\begin{list1}
	\item[] Inżynier, {\bf Informatyka}, Październik 2017 - Luty 2021
	\vspace*{.05in}
	\end{list1}
	

\section{\sc Doświadczenie}

    {\bf od 11.2022} Wykonawca projektu "Surykatka" na {\bf Politechnice Warszawskiej}
    \begin{list2}
    	\item Analiza danych i wizualizacja danych z symulacji dla sił specjalnych
    	\item Technologie: C\#, Unity, OpenGL
    \end{list2}
    
    {\bf od 03.2022} {\bf Politechnika Warszawska}
    \begin{list2}
    	\item Prowadzenie laboratoriów z:
			\begin{list2}
				\item Programowanie w Środowisku Graficznym
				\item Programowanie 3 - Zaawansowane
				\item Systemy Operacyjne 1
				\item Systemy Operacyjne 2
			\end{list2}
		\item Prowadzenie wykładu z Programowanie 3 - Zaawansowane
		\item Prowadzenie ćwiczeń z Modelowanie Geometryczne 2
		\item Złota Kreda 2024
    \end{list2}
    
	{\bf 06.2022 - 11.2022} Projekt Badawczy dla {\bf CD PROJEKT RED}
	\begin{list2}
	    \item Rozwijanie narzędzi i algorytmów dla środowiska wirtualnego
	\end{list2}

	{\bf 07.2020 - 05.2022} Staż/Junior Developer w {\bf Samsung R\&D Poland}
	\begin{list2}
		\item Rozwój platformy testowej dla Androida
		\item Technologie: Java, Kotlin, Android, C\#, Python
	\end{list2}


\section{\sc Umiejętności}
	
	\begin{list2}
		\item Algorytmy i struktury danych
		\item Algorytmy środowiska wirtualnego
		\item Wizualizacja w środowisku graficznym
		\item Projektowanie systemów CAD/CAM
		\item Modelowanie geometryczne
		\item Wysoka znajomość języków {\bf Java} i {\bf C\#}
		\item Doświadczenie z {\bf C++, Python, Linux, OpenGL, Unity} i {\bf Unreal Engine}
	\end{list2}
	
\section{\sc Osiągnięcia Naukowe i Organizacyjne}
	
	\begin{list2}
		\item Artykuł: Płocharski, A., Porter-Sobieraj, J., Lamecki, A., Herman, T., \& Uszakow, A. (2024). Skeleton based tetrahedralization of surface meshes. Computer Aided Geometric Design, 111, 102317.
		\item Wykonawca projektu "Surykatka" na {\bf Politechnice Warszawskiej}
		\item Wykonawca grantu rektorskiego w roku 2021 pt. InnoHaptix
		\item Dzień Popularyzacji Matematyki, 2022-2024
		\item Festiwal Nauki w Warszawie, 2022-2024
		\item "Edukacja w Wirtualnej Rzeczywistości" (POWR.03.01.00-00-T199/18)
	\end{list2}

\end{resume}

\end{document}
